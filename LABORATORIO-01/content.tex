
\section{Introducción}
En este informe, analizaremos la implementación y el rendimiento de dos pares de bucles anidados en C que realizan operaciones en una matriz bidimensional. Estudiaremos cómo el diseño de acceso a memoria afecta al rendimiento debido a la jerarquía de caché de la CPU.

\section{Implementación}
El código proporcionado contiene dos pares de bucles anidados que operan en una matriz bidimensional `A`, un vector `x`, y un vector `y`. El primer par de bucles accede a la matriz `A` en un patrón de acceso de fila principal, mientras que el segundo par de bucles accede a la matriz `A` en un patrón de acceso de columna principal.

\begin{lstlisting}
// Primera pareja de bucles (acceso por filas)
for (i = 0; i < MAX; i++)
    for (j = 0; j < MAX; j++)
        y[i] += A[i][j] * x[j];

// Segunda pareja de bucles (acceso por columnas)
for (j = 0; j < MAX; j++)
    for (i = 0; i < MAX; i++)
        y[i] += A[i][j] * x[j];
\end{lstlisting}

\section{Resultados}
Supongamos que `MAX` es igual a 4 y que la matriz `A` se almacena en la memoria de la siguiente manera:

\begin{table}[h]
\centering
\begin{tabular}{|c|c|c|c|c|}
\hline
\textbf{Línea de Caché} & \textbf{Elementos de A} \\
\hline
0 & A[0][0] A[0][1] A[0][2] A[0][3] \\
1 & A[1][0] A[1][1] A[1][2] A[1][3] \\
2 & A[2][0] A[2][1] A[2][2] A[2][3] \\
3 & A[3][0] A[3][1] A[3][2] A[3][3] \\
\hline
\end{tabular}
\caption{Almacenamiento de elementos de A en caché}
\end{table}

Supongamos que la caché es de mapeo directo y solo puede almacenar ocho elementos de `A`, o dos líneas de caché.

\subsection{Primera Pareja de Bucles (Acceso por Filas)}
Ambos pares de bucles comienzan accediendo a `A[0][0]`, lo que resulta en un fallo de caché. La caché se llena con la primera línea de `A`. Luego, el primer par de bucles accede a elementos contiguos en la primera fila de `A`, que están en caché. Como resultado, se producen cuatro fallos de caché cuando el primer par de bucles accede a las cuatro filas de `A`.

\subsection{Segunda Pareja de Bucles (Acceso por Columnas)}
Para el segundo par de bucles, después de llenar la caché con la primera línea de `A`, se producen tres fallos de caché cuando se accede a las tres columnas restantes. Además, debido a que la caché es pequeña, los accesos a `A[2][0]` y `A[3][0]` provocan que las líneas 0 y 1 de caché sean reemplazadas. Como resultado, el segundo par de bucles experimenta un total de 16 fallos de caché.

\section{Análisis}
La diferencia significativa en el rendimiento entre los dos pares de bucles se debe a cómo acceden a la matriz `A`. El primer par de bucles (acceso por filas) tiene un mejor rendimiento porque accede a los elementos de `A` en un patrón contiguo de fila principal, lo que aprovecha la localidad espacial. El segundo par de bucles (acceso por columnas) tiene un peor rendimiento debido al patrón de acceso no contiguo de columna principal, lo que resulta en una mayor cantidad de fallos de caché.

En la ejecución con un valor de `MAX = 1000`, se confirma la diferencia en el rendimiento, con el primer par de bucles siendo aproximadamente tres veces más rápido que el segundo par.

En conclusión, el diseño del acceso a memoria en el código puede tener un impacto significativo en el rendimiento debido a la jerarquía de caché de la CPU. Es importante optimizar el acceso a datos para aprovechar la localidad espacial y temporal y minimizar los fallos de caché.
